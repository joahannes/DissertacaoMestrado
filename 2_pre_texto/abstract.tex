\begin{abstract}

\begin{center}

Abstract of Dissertation presented to UFPA as a partial fulfillment of the requirements for the degree of Master in Computer Science.

{% Norma - fonte negrito, corpo 16
\bfseries
\LARGE

%Data Dissemination based in Complex Networks metrics for Intelligent Transport Systems

}

\end{center}

\begin{singlespace}
  \noindent Advisor: \ABNTorientadordata \\
  \noindent Co-advisor: \ABNTcoorientadordata\\ %comentar sen�o tiver co-orientador
  \noindent Key words: \ABNTkwumdata; \ABNTkwdoisdata; \ABNTkwtresdata. \\
\end{singlespace}

Services that aim to make the current transportation system more secure, sustainable and efficient form the framework known as Intelligent Transportation Systems (ITSs). Due to the fact that the services require data, communication and processing for operation, Vehicle Ad Hoc Networks (VANETs) exert a strong influence in this context, since it allows direct communication between vehicles and, in addition, data are exchanged and processed between them. Several ITS services require disseminated information among decision-making vehicles. However, such dissemination is a challenging task, due to the specific characteristics of VANETs, such as short-range communication and high node mobility, resulting in constant variations in their topology. In view of the challenges, this paper presents a protocol for data dissemination in urban scenarios that consider complex network metrics in its operation, called DDRX (Data Dissemination based on Complex Network Metrics). The protocol takes advantage of the beacons that are periodically exchanged in the network to collect information from the vehicles and thus build a graph of diameter 2 for local topographic analysis of the network, where the vertices are vehicles and the edges the communication links between neighboring vehicles. With the local graph created, it is possible to identify the best vehicles to continue the dissemination process. Simulation results show that DDRX offers high efficiency in terms of coverage, the number of transmissions, delay, and packet collisions compared to widely known data dissemination protocols. In addition, DDRX offers significant improvements to a distributed traffic management system that needs the disseminated traffic information efficiently, allowing vehicles to spend less time congestion, achieve higher average speeds and have less travel time.

\end{abstract}
